\newlist{legal}{enumerate}{10}
\setlist[legal]{label*=\arabic*.}
\section{Introduction}\label{sec:intro}
\subsection{Concept}

The concept of \emph{anonymous credentials} allows users to prove that their identity satisfies certain properties in an uncorrelated way without revealing other identity details.  The properties can be raw identity attributes  such as
the birth date or the address, or more sophisticated predicates such as ``A is older than 20 years old''.

We assume three parties: \emph{issuer}, \emph{holder}, and \emph{verifier}. From the functional perspective, the issuer gives a credential $C$ based on identity schema $X$, which asserts certain properties $\mathcal{P}$ about $X$, to the holder. The credential consists of attributes represented by integers $m_1, m_2,\ldots, m_l$. The holder then presents $(\mathcal{P},C)$  to the Verifier, which can verify that the issuer has asserted that holder's identity has property $\mathcal{P}$. 

\subsection{Properties}

Credentials are \emph{unforgeable} in the sense that no one can fool the Verifier with a credential not prepared by the issuer.

We say that credentials are  \emph{unlinkable} if it is impossible to correlate the presented credential across multiple presentations. This is implemented by the holder \emph{proving} with a zero-knowledge proof \emph{that he has a credential} rather than showing the credential.

Unlinkability can be simulated by the issuer generating a sufficient number of ordinary unrelated credentials. Also unlinkability can be turned off to make credentials \emph{one-time use} so that second and later presentations are detected.


\subsection{Pseudonyms}
Typically a credential is bound to a certain pseudonym $\mathrm{nym}$. It is supposed that holder has been registered as $\mathrm{nym}$ at the issuer, and communicated (part of) his identity $X$ to him. After  that the issuer can issue a credential that couples $\mathrm{nym}$ and $X$. 

The holder may have a pseudonym at the Verifier, but not necessarily. If there is no pseudonym then the Verifier provides the service to users who did not register. If the pseudonym $\mathrm{nym}_V$ is required, it can be generated from a link secret $m_1$ together with $\mathrm{nym}$ in a way that $\mathrm{nym}$ can not be linked to $\mathrm{nym}_V$. However, holder is supposed to prove that the credential presented was issued to a pseudonym derived from the same link secret as used to produce $\mathrm{nym}_V$.

An identity owner also can create a policy address $I$ that is used for managing agent proving authorization. The address are tied to credentials issued to holders such that agents cannot use these credentials without authorization.

\begin{comment}

\section{Simple example}

 Government (Issuer $G$) issues credentials with age and photo hash. Company ABC-Co (Issuer $A$) issues a credential that includes start-date and employment status, e.g., 'FULL-TIME'.
 
Holder establishes a pseudonym with both issuers. Then he got two credentials independently. After that he proves
to Verifier that he has two credentials such that
\begin{itemize}
\item The same link secret is used in both credentials;
\item The age value in the first credential is over 20;
\item The employment status is 'FULL-TIME'.
\end{itemize}

Steps in Section~\ref{sec:iss-setup} must be executed for each issuer.

Issuer and holder mutually trust each other in submitting values of the right format
during credential's issuance. This trust can be eliminated at the cost of some extra steps.

\subsection{Functionality}

In this description of the protocol, holder has two attributes he hides from issuer -- his link secret and policy address. However, the protocol allows for any number of attributes to be hidden from issuers.
\end{comment}

\section{Generic notation}

Attribute $m$ is a $l_a$-bit unsigned integer\footnote{Technically it is possible to support credentials with different $l_a$, but in Sovrin for simplicity it is set $l_a=256$.}. 


\section{Protocol Overview}

The described protocol supports anonymous credentials given to multiple holders  by various issuers, which are presented to various relying parties.

Various types of anonymous credentials can be supported. In this section, the combination of CL-based credentials~\cite{CamenischL02} and pairing-based revocation~\cite{CamenischKS09} is described.

The simplest credential lifecycle with one credential, single issuer, holder, and verifier is as follows:
\begin{enumerate}
    \item Issuer determines a credential schema $\mathcal{S} $: the type of cryptographic signatures used to sign the credentials, the number $l$ of attributes in a credential, the indices $A_h\subset \{1,2,\ldots,l\}$ of hidden attributes, the public key $P_k$, the non-revocation credential attribute number $l_r$ and non-revocation public key $P_R$ (Section~\ref{sec:iss-setup}). Then he publishes it on the ledger and announces the attribute semantics.
    \item Holder retrieves the credential schema from the ledger and sets the hidden attributes.
    \item Holder requests a credential from issuer. He sends hidden attributes in a blinded form to issuer and agrees on the values of known attributes $A_r=\{1,2,
\ldots,l\}\setminus A_h$.
    \item Issuer returns a credential pair $(C_p, C_{NR})$ to holder. The first credential contains the requested $l$ attributes. The second credential asserts the non-revocation status of the first one. Issuer publishes the non-revoked status of the credential on the ledger.
    \item Holder approaches verifier. Verifier sends the Proof Request $\mathcal{E}$
    to holder. The Proof Request contains the credential schema $\mathcal{S}_E$ and disclosure predicates $\mathcal{D}$. The predicates for attribute $m$ and value $V$ can be of form $m=V$, $m<V$, or $m>V$. Some attributes may be asserted to be the same: $m_i=m_j$.
    \item Holder checks that the credential pair he holds satisfy the schema $\mathcal{S}_E$. 
    He retrieves the non-revocation witness from the ledger.
    \item Holder creates a proof $P$ that he has a non-revoked credential satisfying the proof request $\mathcal{E}$ and sends it to verifier.
    \item Verifier verifies the proof.
\end{enumerate}


If there are multiple issuers, the holder obtains  credentials from them independently. To allow credential chaining, issuers reserve one attribute (usually $m_1$)  for a secret value hidden by holder. Holder is supposed then to set it to the same value in all credentials, 
whereas Relying Parties require them to be equal along all credentials. A proof request should specify then a list of schemas that credentials should satisfy in certain order. 

\section{Schema preparation}




%This section describes how the holder obtains the primary credential and the non-revocation credential from the issuer.
 

Credentials should have limited use to only authorized holder entities called agents. Agents can prove authorization to use a credential by including a policy address $I$ in primary credentials as attribute $m_3$.

\begin{comment}
\subsection{Holder setup}
\begin{enumerate}
    \item Generate a random 256-bit link secret $K$ (possibly the same for all issuers). $m_1 \leftarrow K$ for all credentials.
    \item Generate a random 256-bit policy address $I$ (possibly the same for all issuers). $m_3 \leftarrow I$ for all credentials.
\end{enumerate} 


\subsection{Issuer setup}
\end{comment}
\label{sec:iss-setup}

\subsection{Attributes}\label{sec:setup-attr}
Issuer defines the primary credential schema $\mathcal{S}$ with $l$ attributes $m_1,m_2,\ldots, m_l$ and the set of hidden attributes $A_h \subset \{1,2,\ldots,l\}$. In Sovrin, $m_1$ is reserved for the link secret of the holder, $m_2$ is reserved for the context -- the enumerator for the holders, $m_3$ is reserved for the policy address $I$. By default, $\{1,3\}\subset A_h$ whereas $2\notin A_h$.


Issuer defines the non-revocation credential  with $2$ attributes $m_1,m_2$. In Sovrin, $A_h = \{1\}$ and $m_1$ is reserved for the link secret of the holder, $m_2$ is reserved for the context -- the enumerator for the holders.

%There are attributes known to the issuer and attributes that are known to the holder but are hidden from the issuer.

%Hidden attributes can be included in credentials as blinded values signed by the issuer, or not included and sent to issuers as cryptographic commitments--committed attributes. Primary credential schema $C_s$ attributes are divided into three sets: 
\subsection{Primary Credential Cryptographic Setup}\label{sec:setup-key1}
In Sovrin, issuers use CL-signatures~\cite{ACCUMULATORS} for primary credentials, although other signature types will be supported too.

For the CL-signatures issuer generates:
\begin{enumerate}
    \item Random 1536-bit primes $p',q'$ such that  $p \leftarrow 2p'+1$ and $q \leftarrow 2q'+1$ are primes too. Then compute $n \leftarrow pq$.
    \item A random quadratic residue  $S$ modulo $n$;
    \item Random $x_Z, x_{R_1},\ldots , x_{R_l}\in [2; p'q'-1]$
\end{enumerate}
Issuer computes
\begin{align}
    Z \leftarrow S^{x_Z}\pmod{n};&\quad \{R_i \leftarrow S^{x_{R_i}}\pmod{n}\}_{1\leq i \leq l};
\end{align}
The issuer's public key is $P_k = (n, S,Z,\{R_i\}_{1 \leq i\leq l})$ and the private key is $s_k = (p, q)$.\\
\subsection{Optional: Setup Correctness Proof}\label{sec:setup-proof}
\begin{enumerate}
\item Issuer generates random $\widetilde{x_Z}, \widetilde{x_{R_1}},\ldots,\widetilde{x_{R_l}}\in [2; p'q'-1]$;
\item Computes 
\begin{align}
\widetilde{Z}& \leftarrow S^{\widetilde{x_Z}}\pmod{n};& \{\widetilde{R_i} &
\leftarrow S^{\widetilde{x_{R_i}}}\pmod{n}\}_{1\leq i \leq l};\\
c &\leftarrow H_I(Z||\widetilde{Z}||\{R_i,\widetilde{R_i}\}_{i\leq i \leq l});\\
\widehat{x_Z}& \leftarrow \widetilde{x_Z}+c x_Z \pmod{p'q'};&
\{\widehat{x_{R_i}}&\leftarrow \widetilde{x_{R_i}}+c x_{R_i}\pmod{p'q'}\}_{1\leq i \leq l};
\end{align}
Here $H_I$ is the issuer-defined hash function, by default SHA-256.

\item Proof $\mathcal{P}_I$ of correctness is $(c,\widehat{x_Z},\{\widehat{x_{R_i}}\}_{1 \leq i \leq l})$
\end{enumerate}
\subsection{Non-revocation Credential Cryptographic Setup}
In Sovrin, issuers use CKS accumulator and signatures~\cite{CamenischKS09} to track revocation status of primary credentials, although other signature types will be supported too. Each primary credential is given an index from 1 to $L$.

The CKS  accumulator is used to track revoked primary credentials, or equivalently, their indices. The accumulator contains up to $L$ indices of credentials. If issuer has to issue more credentials, another accumulator is prepared, and so on. Each accumulator $acc_V$ has an identifier $I_{acc}$.

Issuer chooses
\begin{itemize}
    \item Groups $\mathbb{G}_1,\mathbb{G}_2,\mathbb{G}_T$ of
    prime order $q$;
    \item Type-3 pairing operation $e:\, \mathbb{G}_1\times\mathbb{G}_2\rightarrow\mathbb{G}_T$.
    \item Generators: $g$ for $\mathbb{G}_1$, $g'$ for
    $\mathbb{G}_2$.
\end{itemize}

%Typically the triplet $(\mathbb{G}_1,\mathbb{G}_2,\mathbb{G}_T)$ is selected together with a pairing function as only a few combinations admit a suitable pairing. Existing implementations provide just a few possible pairing functions and thus triplets, thus making the group details in fact oblivious to the user. For the sake of curiosity we note that $\mathbb{G}_1,\mathbb{G}_2$ are different groups of elliptic curve points, whereas $\mathbb{G}_T$ is not a curve point group.\\
Issuer:
\begin{legal}
    \item Generates
    \begin{legal}
        \item Random $h,h_0,h_1,h_2,\widetilde{h}\in \mathbb{G}_1$;
        \item Random $u,\widehat{h}\in \mathbb{G}_2$;
        \item Random $sk,x \pmod{q}$.
    \end{legal}
    \item Computes 
\begin{align*}
    pk&\leftarrow g^{sk}; & y&\leftarrow \widehat{h}^x.
\end{align*}
\end{legal}

The revocation public key is
$P_R = (h,h_0,h_1,h_2,\widetilde{h},\widehat{h},u,pk,y)$ and the secret key is $(x,sk)$.
\subsubsection{New Accumulator Setup}
To create a new accumulator $acc_V$, issuer:
\begin{legal}
\item Generates random $\gamma\pmod{q}$.
\item Computes
\begin{legal}
    \item $g_1,g_2,\ldots,g_L,g_{L+2},\ldots,g_{2L}$ where
$g_i = g^{\gamma^i}$. 
    \item $g_1',g_2',\ldots,g_L',g_{L+2}',\ldots,g_{2L}'$ where
$g_i' = g'^{\gamma^i}$. 
    \item $z = (e(g,g'))^{\gamma^{L+1}}$.
\end{legal}
\item Set $V \leftarrow\emptyset$, $\mathrm{acc_V}\leftarrow 1$.
\end{legal}
The accumulator public key is $P_a = (z)$ and secret key is $(\gamma)$.

Issuer publishes $(P_a,V)$ on the ledger. The accumulator identifier is $I_{acc} = z$.

\section{Issuance of Credentials}

\subsection{Holder Setup}

Holder:
\begin{itemize}
    \item Loads credential schema $\mathcal{S}$.
    \item Sets hidden attributes $\{m_i\}_{i \in A_h}$.
    \item Establishes a connection with issuer and gets nonce $n_0$ either from issuer or as a precomputed value. Holder is known to issuer with identifier $\mathcal{H}$.
\end{itemize}

Holder prepares data for primary credential:
\begin{enumerate}
%\item Generate random 256-bit $\{r_i\}_{i \in A_h}$. %FOR COMMITMENTS
\item Generate random 3152-bit $v'$.
\item Generate random 593-bit $\{\widetilde{m_i}
%, \widetilde{r_i}  %FOR COMMITMENTS
\}_{i \in A_h}$, and random 3488-bit $\widetilde{v'}$.
\item Compute taking $S,Z,R_i$ from $P_k$:
\begin{align}
U& \leftarrow (S^{v'}) \prod_{i \in A_h}{R_i^{m_i}} \pmod{n};
%\\ \{C_i& \leftarrow Z^{m_i}S^{r_i} \pmod{n}\}_{i \in A_h};
%FOR COMMITMENTS
\end{align}
\item For proving correctness of $U$, compute
\begin{align}
\widetilde{U}&\leftarrow (S^{\widetilde{v'}}) \prod_{i \in A_h}{R_i^{\widetilde{m_i}}}\pmod{n};
&%\{\widetilde{C_i}& \leftarrow Z^{\widetilde{m_i}}S^{\widetilde{r_i}}\pmod{n}\}_{i \in A_h};&
%FOR COMMITMENTS
\\
c\leftarrow& H(U||\widetilde{U}||
%\{C_i, \widetilde{C_i} \}_{i \in A_h}|| %FOR COMMITMENTS
n_0);&
\widehat{v'}&\leftarrow \widetilde{v'} + c v';&\\
\{\widehat{m_i}&\leftarrow \widetilde{m_i} + c m_i\}_{i \in A_h};&
%\{\widehat{r_i}& \leftarrow \widetilde{r_i} + c r_i\}_{i \in A_h}
%FOR COMMITMENTS
\end{align}
\item Generate random 80-bit nonce $n_1$
\item Send $
\{U, c,\widehat{v'}, \{
%C_i, %FOR COMMITMENTS
\widehat{m_i}
%, \widehat{r_i} %FOR COMMITMENTS
\}_{i \in A_h}, n_1\}$ to the issuer.
\end{enumerate}
Holder prepares for non-revocation credential:
\begin{enumerate}
    \item Load issuer's revocation key $P_R$ and generate random $s'\bmod{q}$.
     \item Compute $U_R \leftarrow h_2^{s'}$
    taking $h_2$ from $P_R$.
    \item Send $U_R$ to the issuer.
    \item For proving correctness of $U_R$
    \begin{itemize}
    \item generate random $\widetilde{s'}\bmod{q}$ and compute $\widetilde{U_R} \leftarrow {h_2}^{\widetilde{s'}}$
    \item Compute above challenge $c$ as $c\leftarrow H(U||\widetilde{U}||U_R||\widetilde{U_R}||n_0)$ instead of $c\leftarrow H(U||\widetilde{U}||n_0)$
    \item Compute $\widehat{s'}\leftarrow \widetilde{s'} + c s'$
    \item Send $c$ and $\widehat{s'}$ to issuer
    \end{itemize}
\end{enumerate}


\subsubsection{Optional: Issuer Proof of Setup Correctness}

To verify the proof $\mathcal{P}_i$ of correctness, holder
computes
$$
\widehat{Z} \leftarrow Z^{-c} S^{\widehat{x_Z}}\pmod{n} ;\quad  \{\widehat{R_i} 
\leftarrow R_i^{-c} S^{\widehat{x_{R_i}}}\pmod{n}\}_{1\leq i \leq l};
$$
and verifies 
$$
c =  H_I(Z||\widehat{Z}||\{R_i,\widehat{R_i}\}_{1\leq i \leq l})
$$.
%For the new user issuer selects the accumulator index $A_{R_i}$ and the user index $i$ so that $(A_{R_i},i)$ is unique.  


\subsection{Primary Credential Issuance}
\begin{comment}
\begin{enumerate}
    \item Retrieve the current  value $\mathrm{acc}$ for accumulator $A_{R_i}$ and the  set $V$ of issued and non-revoked credential numbers.
    \item $H()$ is a hash function where the output length is 256 bits.
    \item $||$ denotes concatenation.
    \item $U_i$ is the identifier of the user $i$.
    \item Compute $$
\overline{S} = A_{R_i}||U_i,\quad H_{\overline{S}} = H(\overline{S})
$$ 
and sets $m_2 = H_{\overline{S}}$.
    \item Create 256-bit integer attributes $\{m_i\}_{i \in A_r}$ for the holder.
    \item Generate 80-bit nonce $n_0$ and send to the holder.
\end{enumerate}
Holder:

\end{comment}


Issuer verifies the correctness of holder's input:
\begin{enumerate}
\item Compute
\begin{align}
\widehat{U}& \leftarrow (U^{-c}) \prod_{i \in A_h}{R_i^{\widehat{m_i}}}(S^{\widehat{v'}})\pmod{n};
%\\
%\{\widehat{C_i}& \leftarrow {C_i}^{-c}Z^{\widehat{m_i}}S^{\widehat{r_i}}\pmod{n}\}_{i \in A_h}
\end{align}
\item Verify 
$c= H(U||\widehat{U}||
%\{C_i,\widehat{C_i}\}_{i \in \mathcal{A}_c}||
n_0)$
\item Verify that $\widehat{v'}$ is a 673-bit number, $\{\widehat{m_i}, \widehat{r_i}\}_{i \in \mathcal{A}_c}$ are 594-bit numbers.
\item If a revocable credential is requested
\begin{itemize}
\item Compute $\widehat{U_R} = {U_R}^{-c}{h_2}^{\widehat{s'}}$
\item Verify that $c$ equals $H(U||\widehat{U}||U_R||\widehat{U_R}||n_0)$ instead of $H(U||\widehat{U}||n_0)$
\end{itemize}
\end{enumerate}
Issuer prepare the credential:
\begin{enumerate}
\item Assigns index $i<L$ to holder, which is one of not yet taken indices for the issuer's current accumulator $acc_V$. Compute $m_2\leftarrow H(i||\mathcal{H})$ and store information about holder and the value $i$ in a local database.
\item Set, possibly in agreement with holder, the values of disclosed attributes, i.e. with indices from $A_r$.
\item Generate random 2724-bit number $v''$ with most significant bit equal 1 and random prime  $e$ such that
\begin{equation}\label{eq:e}
2^{596}\leq e \leq 2^{596}+ 2^{119}.
\end{equation}
\item Compute
\begin{align}
Q& \leftarrow \frac{Z}{U S^{v''} \prod_{i\in \mathcal{A}_k}{R_i^{m_i}}\pmod{n}};\\
A& \leftarrow Q^{e^{-1}\pmod{p'q'}}\pmod{n};
\end{align}
\item Generate random $r < p'q'$;
\item Compute
\begin{align}
\widehat{A}&\leftarrow Q^r\pmod{n};\\
c'&\leftarrow H(Q||A||\widehat{A}||n_1);\\
s_e&\leftarrow r - c'e^{-1}\pmod{p'q'};
\end{align}
\item Send the primary pre-credential  $(\{m_i\}_{i\in A_r},A,e,v'',s_e,c')$ to the holder.
\end{enumerate}

\subsection{Non-revocation Credential Issuance}

%We assume that the attribute $m_2$ is used to enumerate holders by issuer (details are irrelevant for revocation).\newline\newline
Issuer:
\begin{enumerate}
    \item Generate random numbers $s'',c\bmod{q}$.
    \item Take $m_2$ from the primary 
    credential he is preparing for holder.
    \item Take $acc_V$ as the accumulator value for which index $i$ was taken. Retrieve current set of non-revoked indices $V$.
    \item Compute:
\begin{align}
\sigma &\leftarrow \left( h_0 h_1^{m_2}\cdot U_R\cdot  g_i\cdot  h_2^{s''}\right)^{\frac{1}{x+c}};&
w &\leftarrow \prod_{j\in V}g_{L+1-j+i}';\\
\sigma_i &\leftarrow g'^{1/(sk+\gamma^i)};&
u_i &\leftarrow u^{\gamma^i};\\
acc_V&\leftarrow acc_V\cdot g'_{L+1-i};&
V&\leftarrow V\cup\{i\};\\
\mathrm{wit}_i&\leftarrow\{\sigma_i,u_i,g_i,w,V\}.
\end{align}
\item Send the non-revocation pre-credential  $(I_{acc},\sigma,c,s'',\mathrm{wit}_i,g_i,g_i',i)$ to holder.
\item  Publish updated $V, acc_V$ on the ledger.
\end{enumerate}


\subsection{Storing Credentials}
Let $C_s$ be the index set of all attributes. Holder works with the primary pre-credential:
\begin{enumerate}
\item Compute $v \leftarrow v'+v''$.
\item Verify $e$ is prime and satisfies Eq.~\eqref{eq:e}.
\item Compute
\begin{align}
Q\leftarrow \frac{Z}{S^v\prod_{i \in C_s}R_i^{m_i}}\pmod{n};
\end{align}
\item Verify $Q = A^e\pmod{n}$
\item Compute
\footnote{We have removed factor $S^{v's_e}$ here from computing of $\widehat{A}$ as it seems to be a typo in the Idemix spec.}
\begin{align}
\widehat{A}\leftarrow A^{c'+s_e\cdot e} \pmod{n}.
\end{align}
\item Verify $c'=H(Q||A||\widehat{A}||n_2).$
\item Store \emph{primary credential} $C_p=(\{m_i\}_{i \in C_s},A,e,v)$.
\end{enumerate}
Holder takes the non-revocation pre-credential $(I_{acc},\sigma,c,s'',\mathrm{wit}_i,g_i,g_i',i)$ computes $s \leftarrow s'+s''$ and stores the non-revocation credential $C_{NR}\leftarrow(I_{acc},\sigma,c,s,\mathrm{wit}_i,g_i,g_i',i)$.
\subsection{Non revocation proof of correctness} Holder computes
\begin{align}
\frac{e(g_i,acc_V)}{e(g,w)} &\overset{\text{?}}{=} z;\\
e(pk\cdot g_i, \sigma_i) &\overset{\text{?}}{=} e(g,g');\\
e(\sigma,y\cdot \widehat{h}^c)& \overset{\text{?}}{=} e(h_0 \cdot h_1^{m_2}h_2^{s}g_i,\widehat{h}).
\end{align}
    

\section{Revocation}
Issuer identifies a credential to be revoked in the database and retrieves its index $i$, the  accumulator value $acc_V$, and valid index set $V$. Then he proceeds:
\begin{enumerate}
\item Set $V\leftarrow V\setminus\{i\}$;
\item Compute $acc_V \leftarrow acc_V/g'_{L+1-i}$.
\item Publish $\{V,acc_V\}$.
\end{enumerate}
    
\section{Presentation}

\subsection{Proof Request}

Verifier sends a proof request, where it specifies the ordered set of $d$ credential schemas
$\{\mathcal{S}_1,\mathcal{S}_2,\ldots,\mathcal{S}_d\}$, so that the holder should provide a set of $d$ credential pairs $(C_p,C_{NR})$ that correspond to these schemas.

Let credentials in these schemas contain $X$ attributes in total. Suppose that the request makes to open $x_1$ attributes, makes to prove $x_2$ equalities $m_i=m_j$ (from possibly distinct schemas) and makes to prove $x_3$ predicates of form  $m_i >\leq \geq<z$. Then effectively $X-x_1$ attributes are unknown (denote them $A_h$), which form $x_4=(X-x_1-x_2)$ equivalence classes. Let $\phi$ map $A_h$ to $\{1,2,\ldots,x_4\}$ according to this equivalence.  Let $A_r$ denote the set of indices of $x_1$ attributes that are disclosed.

The proof request also specifies $A_h,\phi,A_r$ and the set $\mathcal{D}$ of predicates. Along with a proof request, Verifier also generates and sends 80-bit nonce $n_1$.

\subsection{Proof Preparation}
Holder prepares all credential pairs $(C_p,C_{NR})$ to submit:
\begin{enumerate}
\item Generates $x_4$ random 592-bit values $\widetilde{y_1},\widetilde{y_2},
\ldots,\widetilde{y_{x_4}}$ and set $\widetilde{m_j} \leftarrow \widetilde{y_{\phi(j)}} $ for  $j \in \mathcal{A}_{h}$. 
\item
 Create empty sets $\mathcal{T}$ and $\mathcal{C}$.
\item For all credential pairs $(C_p,C_{NR})$ executes Section~\ref{sec:prepare}. 
\item Executes Section~\ref{sec:hash} once. 
\item For all credential pairs $(C_p,C_{NR})$ executes Section~\ref{sec:final}.
\item Executes Section~\ref{sec:final} once.
\end{enumerate}
Verifier:
\begin{enumerate}
\item For all credential pairs $(C_p,C_{NR})$ executes Section~\ref{sec:verify}.
\item Executes Section~\ref{sec:finalhash} once.
\end{enumerate}

\label{sec:prepare}

\textbf{Non-revocation proof}
Holder:
\begin{enumerate}
\item Load issuer's public revocation key $P_R = (h,h_1,h_2,\widetilde{h},\widehat{h},u,pk,y)$.
\item Load the non-revocation credential $C_{NR}\leftarrow(I_{acc},\sigma,c,s,\mathrm{wit}_i,g_i,g_i',i)$;
\item Obtain recent $V,\mathrm{acc_V}$ (from Verifier, Sovrin link, or elsewhere).
\item Update $C_{NR}$:
\begin{align*}
w&\leftarrow w\cdot \frac{\prod_{j\in V\setminus V_{old}}g'_{L+1-j+i}}{\prod_{j\in V_{old}\setminus V}g'_{L+1-j+i}};\\
V_{old}&\leftarrow V.
\end{align*}
Here $ V_{old}$ is taken from $\mathrm{wit}_i$ and updated there.

\item Select random $\rho,\rho',r,r',r'',r''',o,o'\bmod{q}$;
\item Compute
\begin{align}
E &\leftarrow h^{\rho} \widetilde{h}^o &
D & \leftarrow g^r\widetilde{h}^{o'};\\
A &\leftarrow \sigma \widetilde{h}^{\rho}&
\mathcal{G} &\leftarrow g_i\widetilde{h}^r;\\
\mathcal{W} &\leftarrow w\widehat{h}^{r'} &
\mathcal{S}&\leftarrow \sigma_i \widehat{h}^{r''}\\
\mathcal{U}&\leftarrow u_i \widehat{h}^{r'''}
\end{align}
and adds these values to $\mathcal{C}$.
\item Compute 
\begin{align}
m&\leftarrow \rho \cdot c \mod{q}; & t&\leftarrow o\cdot c \mod{q};\\
m'&\leftarrow r\cdot r''\mod{q}; & t'&\leftarrow o'\cdot r'' \mod{q};
\end{align}
and adds these values to $\mathcal{C}$.
\item Generate random $\widetilde{\rho},\widetilde{o},\widetilde{o'},\widetilde{c},
\widetilde{m},\widetilde{m'},\widetilde{t},\widetilde{t'},
\widetilde{m_2},\widetilde{s},
\widetilde{r},\widetilde{r'},\widetilde{r''},\widetilde{r'''},
\bmod{q}$.

\item Compute 
\begin{align}
\overline{T_1}&\leftarrow h^{\widetilde{\rho}} \widetilde{h}^{\widetilde{o}} &
\overline{T_2}&\leftarrow E^{\widetilde{c}}h^{-\widetilde{m}}\widetilde{h}^{-\widetilde{t}}
\end{align}
\begin{equation}
\overline{T_3}\leftarrow e(A,\widehat{h})^{\widetilde{c}}\cdot e(\widetilde{h},\widehat{h})^{\widetilde{r}}\cdot
e(\widetilde{h},y)^{-\widetilde{\rho}}\cdot
e(\widetilde{h},\widehat{h})^{-\widetilde{m}}\cdot
e(h_1,\widehat{h})^{-\widetilde{m_2}}\cdot e(h_2,\widehat{h})^{-\widetilde{s}}\\
\end{equation}
\begin{align}
\overline{T_4}&\leftarrow e(\widetilde{h},\mathrm{acc_V})^{\widetilde{r}}\cdot
e(1/g,\widehat{h})^{\widetilde{r'}}&
\overline{T_5}&\leftarrow g^{\widetilde{r}}\widetilde{h}^{\widetilde{o'}}\\
\overline{T_6}&\leftarrow D^{\widetilde{r''}}g^{-\widetilde{m'}}
\widetilde{h}^{-\widetilde{t'}}&
\overline{T_7}&\leftarrow e(pk\cdot \mathcal{G},\widehat{h})^{\widetilde{r''}}\cdot 
e(\widetilde{h},\widehat{h})^{-\widetilde{m'}}\cdot
e(\widetilde{h},\mathcal{S})^{\widetilde{r}}\\
\overline{T_8}&\leftarrow e(\widetilde{h},u)^{\widetilde{r}}
\cdot e(1/g,\widehat{h})^{\widetilde{r'''}}
\end{align}
and add these values to $\mathcal{T}$.
\end{enumerate}
\textbf{Validity proof}\\
\\Holder:
\begin{legal}
\item Generate a random 592-bit number $\widetilde{m_j}$ for each $j \in \mathcal{A}_h$.
\item For each credential $C_p = (\{m_j\},A,e,v)$ and issuer's
public key $pk_I$:
\begin{legal}
\item Choose random 3152-bit $r$.
\item Take $n,S$ from $pk_I$ compute 
\begin{equation}\label{eq:aprime}
A' \leftarrow A S^{r}\pmod{n}
\text{ and } v' \leftarrow v - e\cdot r\text{ as integers};
\end{equation}
and add to $\mathcal{C}$.
\item Compute $e' \leftarrow e - 2^{596}$.
\item Generate random 456-bit number $\widetilde{e}$.
\item Generate random 3748-bit number $\widetilde{v}$.
\item Compute 
\begin{align}
T \leftarrow (A')^{\widetilde{e}}\left(\prod_{j\in \mathcal{A}_h} R_j^{\widetilde{m_j}}\right)(S^{\widetilde{v}})\pmod{n}
\end{align}
and add to $\mathcal{T}$.
\end{legal}
\item Load $Z,S$ from issuer's public key.
\item For each predicate $p$ where the operator $*$ is one of $>, \geq, <, \leq$.
\begin{legal}
\item Calculate $\Delta$ such that:
$$
\Delta \leftarrow \begin{cases}
z_j-m_j; & \mbox{if } * \equiv\ \leq\\
z_j-m_j-1; & \mbox{if } * \equiv\ <\\
m_j-z_j; & \mbox{if } * \equiv\ \geq\\
m_j-z_j-1; & \mbox{if } * \equiv\ >
\end{cases}
$$
\item Calculate $a$ such that:
$$
a \leftarrow \begin{cases}
-1 & \mbox{if } * \equiv \leq or <\\
1  & \mbox{if } * \equiv \geq or >
\end{cases}
$$
 \item Find (possibly by exhaustive search) $u_1, u_2,u_3, u_4$ such that:
 \begin{align}
\Delta = (u_1)^2+ (u_2)^2+ (u_3)^2+ (u_4)^2
\end{align}
\item Generate random 2128-bit numbers $r_1,r_2,r_3,r_4, r_{\Delta}$.
\item Compute
\begin{align}
\{T_i &\leftarrow Z^{u_i}S^{r_i} \pmod{n}\}_{1 \leq i \leq 4};\\
T_{\Delta} &\leftarrow  Z^{\Delta}S^{r_{\Delta}} \pmod{n};
\end{align}
and add these values to $\mathcal{C}$ in the order $T_1,T_2,T_3,T_4,T_{\Delta}$.
\item Generate random 592-bit numbers $\widetilde{u_1},\widetilde{u_2},\widetilde{u_3},\widetilde{u_4}$.
\item Generate random 672-bit numbers $\widetilde{r_1},\widetilde{r_2},\widetilde{r_3},\widetilde{r_4},\widetilde{r_{\Delta}}$.
\item Generate random 2787-bit number $\widetilde{\alpha}$
\item Compute
\begin{align}
\{\overline{T_i} &\leftarrow Z^{\widetilde{u_i}}S^{\widetilde{r_i}}\pmod{n}\}_{1 \leq i \leq 4};\\
\overline{T_{\Delta}} &\leftarrow  Z^{\widetilde{m_j}}S^{a \widetilde{r_{\Delta}}} \pmod{n};\\
Q &\leftarrow (S^{\widetilde{\alpha}})\prod_{i=1}^{4}{T_i^{\widetilde{u_i}}}\pmod{n};
\end{align}
and add these values to $\mathcal{T}$ in the order $\overline{T_1},\overline{T_2},\overline{T_3},\overline{T_4}, \overline{T_{\Delta}},Q$.
\end{legal}
\end{legal}
\subsubsection{Hashing}\label{sec:hash}

Holder computes challenge hash
\begin{align}
c_H \leftarrow H(\mathcal{T},\mathcal{C},n_1);
\end{align}
and sends $c_H$ to Verifier. 

\subsubsection{Final preparation}\label{sec:final}
Holder:
\begin{enumerate}
\item For non-revocation credential $C_{NR}$ compute:
\begin{align*}
\widehat{\rho} &\leftarrow \widetilde{\rho} - c_H\rho\bmod{q} &
\widehat{o} &\leftarrow \widetilde{o} - c_H\cdot o\bmod{q}\\
\widehat{c} &\leftarrow \widetilde{c} - c_H\cdot c\bmod{q} &
\widehat{o'} &\leftarrow \widetilde{o'} - c_H\cdot o'\bmod{q}\\
\widehat{m} &\leftarrow \widetilde{m} - c_H m\bmod{q} &
\widehat{m'} &\leftarrow \widetilde{m'} - c_H m'\bmod{q}\\
\widehat{t} &\leftarrow \widetilde{t} - c_H t\bmod{q} &
\widehat{t'} &\leftarrow \widetilde{t'} - c_H t'\bmod{q}\\
\widehat{m_2} &\leftarrow \widetilde{m_2} - c_H m_2\bmod{q} &
\widehat{s} &\leftarrow \widetilde{s} - c_H s\bmod{q}\\
\widehat{r} &\leftarrow \widetilde{r} - c_H r\bmod{q} &
\widehat{r'} &\leftarrow \widetilde{r'} - c_H r'\bmod{q}\\
\widehat{r''} &\leftarrow \widetilde{r''} - c_H r''\bmod{q} &
\widehat{r'''} &\leftarrow \widetilde{r'''} - c_H r'''\bmod{q}.
\end{align*}
and add them to $\mathcal{X}$.
\item For primary credential $C_p$ compute:
\begin{align}
\widehat{e}& \leftarrow \widetilde{e}+c_H e';\\
\widehat{v}& \leftarrow \widetilde{v}+c_H v';\\
\{\widehat{m}_j& \leftarrow \widetilde{m_j} + c_H m_j\}_{j \in \mathcal{A}_h};
\end{align}
The values $Pr_C=(\widehat{e},\widehat{v},\{\widehat{m_j}\}_{j \in \mathcal{A}_h},A')$ are the \emph{sub-proof}
for credential $C_p$.
\item For each predicate $p$ compute:
\begin{align}
\{\widehat{u_i}& \leftarrow \widetilde{u_i}+c_H u_i\}_{1\leq i \leq 4};\\
\{\widehat{r_i}& \leftarrow \widetilde{r_i}+c_H r_i\}_{1\leq i \leq 4};\\
\widehat{r_{\Delta}}& \leftarrow \widetilde{r_{\Delta}}+c_H r_{\Delta};\\
\widehat{\alpha}& \leftarrow \widetilde{\alpha}+c_H (r_{\Delta}- u_1r_1 - u_2r_2 - u_3r_3 - u_4r_4); 
\end{align}
The values $Pr_p =( \{\widehat{u_i}\}, \{\widehat{r_i}\},\widehat{r_{\Delta}},\widehat{\alpha},\widehat{m_j})$ are the sub-proof for predicate $p$.
\end{enumerate}

\subsubsection{Sending}\label{sec:send}
 Holder sends $(c_H,\mathcal{X},\{Pr_C\},\{Pr_p\},\mathcal{C})$  to the Verifier.

\subsection{Verification}\label{sec:verify}
For the credential pair $(C_p,C_{NR})$, Verifier retrieves relevant variables from $\mathcal{X},\{Pr_C\},\{Pr_p\},\mathcal{C}$. 

\subsubsection{Non-revocation check}
 
Verifier computes
\begin{align}
\widehat{T_1}&\leftarrow E^{c_H}\cdot h^{\widehat{\rho}} \cdot \widetilde{h}^{\widehat{o}} &
\widehat{T_2}&\leftarrow E^{\widehat{c}}\cdot h^{-\widehat{m}}\cdot\widetilde{h}^{-\widehat{t}}
\end{align}
\begin{equation}
\widehat{T_3}\leftarrow\left(\frac{e(h_0\mathcal{G},\widehat{h})}{e(A,y)} \right)^{c_H} \cdot e(A,\widehat{h})^{\widehat{c}}\cdot e(\widetilde{h},\widehat{h})^{\widehat{r}}\cdot
e(\widetilde{h},y)^{-\widehat{\rho}}\cdot
e(\widetilde{h},\widehat{h})^{-\widehat{m}}\cdot
e(h_1,\widehat{h})^{-\widehat{m_2}}\cdot e(h_2,\widehat{h})^{-\widehat{s}}\\
\end{equation}
\begin{align}
\widehat{T_4}&\leftarrow\left(\frac{e(\mathcal{G},\mathrm{acc_V})}{e(g,\mathcal{W})z}\right)^{c_H} \cdot e(\widetilde{h},\mathrm{acc_V})^{\widehat{r}}\cdot
e(1/g,\widehat{h})^{\widehat{r'}}
&
\widehat{T_5}&\leftarrow D^{c_H}\cdot g^{\widehat{r}}\widetilde{h}^{\widehat{o'}}\\
\widehat{T_6}&\leftarrow  D^{\widehat{r''}}\cdot g^{-\widehat{m'}}
\widetilde{h}^{-\widehat{t'}}&
\widehat{T_7}&\leftarrow
\left(\frac{e(pk\cdot\mathcal{G},\mathcal{S})}{e(g,g')}\right)^{c_H}\cdot e(pk\cdot \mathcal{G},\widehat{h})^{\widehat{r''}}\cdot 
e(\widetilde{h},\widehat{h})^{-\widehat{m'}}\cdot
e(\widetilde{h},\mathcal{S})^{\widehat{r}}\\
\widehat{T_8}&\leftarrow \left(\frac{e(\mathcal{G},u)}{e(g,\mathcal{U})}\right)^{c_H}\cdot e(\widetilde{h},u)^{\widehat{r}}
\cdot e(1/g,\widehat{h})^{\widehat{r'''}}
\end{align}
and adds these values to $\widehat{T}$.

\subsubsection{Validity}
Verifier uses all issuer public key $pk_I$ involved into the credential generation and  the received $(c,\widehat{e},\widehat{v},\{\widehat{m_j}\},A')$. He also uses revealed 
$\{m_j\}_{j \in \mathcal{A}_r}$. He initiates $\widehat{\mathcal{T}}$ as empty set.


\begin{legal}
\item For each credential $C_p$, take each sub-proof $Pr_C$ and compute 
\begin{equation}\label{eq:that}
 \widehat{T} \leftarrow \left(
    \frac{Z}
    { \left(
        \prod_{j \in \mathcal{A}_r}{R_j}^{m_j}
    \right)
    (A')^{2^{596}}
    }\right)^{-c}
    (A')^{\widehat{e}}
    \left(\prod_{j\in (\mathcal{A}_h)}{R_j}^{\widehat{m_j}}\right)
    (S^{\widehat{v}})\pmod{n}.
\end{equation}
Add $\widehat{T}$ to $\widehat{\mathcal{T}}$.
\item For each predicate $p$:
$$
\Delta' \leftarrow \begin{cases}
z_j; & \mbox{if } * \equiv\ \leq\\
z_j-1; & \mbox{if } * \equiv\ <\\
z_j; & \mbox{if } * \equiv\ \geq\\
z_j+1; & \mbox{if } * \equiv\ >
\end{cases}
$$
$$
a \leftarrow \begin{cases}
-1 & \mbox{if } * \equiv \leq or <\\
1  & \mbox{if } * \equiv \geq or >
\end{cases}
$$
\begin{legal}
\item Using $Pr_p$ and $\mathcal{C}$ compute 
\begin{align}
\{\widehat{T_i} &\leftarrow T_i^{-c}Z^{\widehat{u_i}} S^{\widehat{r_i}}\pmod{n}\}_{1\leq i \leq 4};\label{eq:pr2}\\
\widehat{T_{\Delta}} &\leftarrow \left(T_{\Delta}^{a}Z^{\Delta'}\right)^{-c}Z^{\widehat{m_j}}S^{a\widehat{r_{\Delta}}}\pmod{n};\label{eq:pr1}\\
\widehat{Q}&\leftarrow (T_{\Delta}^{-c})\prod_{i=1}^{4}T_i^{\widehat{u_i}}(S^{\widehat{\alpha}})\pmod{n}\label{eq:pr3},
\end{align}
and add these values to  $\widehat{\mathcal{T}}$ in the order $\widehat{T_1},\widehat{T_2} ,\widehat{T_3},\widehat{T_4},\widehat{T_{\Delta}},\widehat{Q}$.
\end{legal}
\end{legal}
\subsubsection{Final hashing}\label{sec:finalhash}
\begin{enumerate}
\item Verifier computes 
$$
\widehat{c_H}\leftarrow H(\widehat{\mathcal{T}},\mathcal{C},n_1).
$$
\item If $c_H=\widehat{c_H}$ output VERIFIED else FAIL.
\end{enumerate}

 
 \ifdef{\fullpaper}{
 \section{Changelog}
 \subsection{20 Jun 2019 (version 0.5)}
 \begin{itemize}
 \item Holder's proof of correctness for revocation secret
 \item Fixing several typos
 \end{itemize}
 \subsection{9 Feb 2018 (version 0.4)}
 Formatting and updates for committed attributes
 \subsection{7 Feb 2018 (version 0.3)}
 Type-3-pairing-based revocation added.
 \subsection{7 Feb 2018 (version 0.21)}
 \begin{itemize}
 \item $c$ changed to $-c$ in Section 5, item 1.0.1.
 \item Factor $S^{v's_e}$ is removed from item 3.2.0.
 \end{itemize}
 \subsection{13 July 2017}
 Added:
 \begin{itemize}    
 \item Proof of correctness for issuer's setup in Section~\ref{sec:iss-setup};
 \item Verification of correctness of setup: steps 1.0.1, 1.0.2;
 \item Proof of correctness for holder's blinded attributes: steps 1.3.1, 1.3.2, 1.4;
 \item Verification holder's proof of correctness: steps 2.0.1, 2.0.2;
 \item Issuer sends all $m_i$ in step 2.4.
 \item Proof of correctness for issuer's signature: steps 2.2.1, 2.2.2, 2.2.3.
 \item  Verification of correctness of signature: steps 3.1.0, 3.1.1, 3.1.2, 3.2.0, 3.2.1.
 \end{itemize}
 }{}